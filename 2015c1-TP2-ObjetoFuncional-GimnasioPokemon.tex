\documentclass[spanish,a4paper]{article}

\newcommand{\version}{Trabajo Práctico} % final, interna o con soluciones (ojo escribirlo 'tal cual').
\newcommand{\cuatrimestre}{1er cuatrimestre 2015} 
\newcommand{\nombreTP}{Gimnasio Pokemon}
\newcommand{\urlReferencia}{\url{http://es.pokemon.wikia.com/wiki/Pokémon}}
%%!TEX encoding = UTF-8 Unicode 
\usepackage[utf8]{inputenc}
\usepackage{amstext}
\usepackage{ifthen}
\usepackage{graphicx}
\usepackage{array}
\usepackage{xspace}
\usepackage{color}
\usepackage{fancyvrb}
\usepackage{version}
\usepackage{theorem}
\usepackage{listings}
\usepackage{hyperref}

\usepackage[spanish]{babel}

\textwidth 17.5cm
\textheight 26cm
\oddsidemargin -1cm
%\evensidemargin -.3cm
\topmargin -1cm
\headheight 0cm
\headsep 0cm

%\textwidth 16cm
%\textheight 25cm
%\oddsidemargin -.64cm
%\evensidemargin -.64cm
%\topmargin -1.5cm
%\headheight 0cm
%\headsep 0cm
\definecolor{gray}{rgb}{0.5,0.5,0.5}

% "define" Scala
\lstdefinelanguage{scala}{
  morekeywords={abstract,case,catch,class,def,%
    do,else,extends,false,final,finally,%
    for,if,implicit,import,match,mixin,%
    new,null,object,override,package,%
    private,protected,requires,return,sealed,%
    super,this,throw,trait,true,try,%
    type,val,var,while,with,yield},
  otherkeywords={=>,<-,<\%,<:,>:,\#,@},
  sensitive=true,
  morecomment=[l]{//},
  morecomment=[n]{/*}{*/},
  morestring=[b]",
  morestring=[b]',
  morestring=[b]"""
}

\lstset{ %
  backgroundcolor=\color{white},  % choose the background color. You must add \usepackage{color}
  showspaces=false,               % show spaces adding particular underscores
  showstringspaces=false,         % underline spaces within strings
  showtabs=false,                 % show tabs within strings adding particular underscores
  frame=single,	                % adds a frame around the code
  tabsize=2,	                % sets default tabsize to 2 spaces
  captionpos=true,                   % sets the caption-position to bottom
  breaklines=true,                % sets automatic line breaking
  breakatwhitespace=true,        % sets if automatic breaks should only happen at whitespace
  escapeinside={\%*}{*)/},         % if you want to add a comment within your code
  aboveskip=3mm,
  belowskip=3mm,
  columns=flexible,
  basicstyle=\small\ttfamily,
  numberstyle=\tiny\color{gray},
  keywordstyle=\bfseries,
  commentstyle=\color{gray},
  stringstyle=\color{gray},
}

\lstdefinelanguage{smalltalk}{
  morekeywords={self,super},
  otherkeywords={:=,^,\[,\],>>,\(,\),\;},
  sensitive=true,
  morecomment=[n]{"},
  morestring=[b]'
}

\definecolor{dkgreen}{rgb}{0,0.6,0}
\definecolor{gray}{rgb}{0.5,0.5,0.5}
\definecolor{mauve}{rgb}{0.58,0,0.82}
 
% Default settings for code listings
\lstset{frame=tb,
  language=scala,
  aboveskip=3mm,
  belowskip=3mm,
  showstringspaces=false,
  columns=flexible,
  basicstyle={\small\ttfamily},
  numbers=none,
  numberstyle=\tiny\color{gray},
  keywordstyle=\color{blue},
  commentstyle=\color{dkgreen},
  stringstyle=\color{mauve},
  frame=single,
  breaklines=true,
  breakatwhitespace=true
  tabsize=3
}

\renewcommand{\thepage}{}
\newcommand{\ignore}[1]{}
\newcommand{\mat}[1]{\ensuremath{#1}}

\newenvironment{display}
   {\begin{list}{}{\setlength{\topsep}{0cm}
                   \setlength{\leftmargin}{0cm}
                  }}{\end{list}}

\newcommand{\textfol}[1]{\ensuremath{\textsf{#1}}}
\newcommand{\textcode}[1]{{\normalfont{\texttt{#1}}}}
\newcommand{\propername}[1]{\textsc{#1}}

%\renewcommand{\labelenumi}{\textbf{\alph{enumi})}}
\renewcommand{\labelenumii}{\textbf{\alph{enumii})}}

\newcommand{\pred}[2]{\ensuremath{\text{\textbf{#1}}(#2)}}
\newcommand{\predName}[1]{\ensuremath{\text{\textbf{#1}}}}
\newcommand{\var}[1]{\ensuremath{#1}}
\newcommand{\code}[1]{\texttt{#1}}

\newcommand{\paratodo}[2]{\ensuremath{(\forall{#1})#2}}
\newcommand{\existe}[2]{\ensuremath{(\exists{#1})#2}}
\newcommand{\implicasp}{\ensuremath{\Rightarrow}}
\newcommand{\ysp}{\ensuremath{\,\wedge\,}}
\newcommand{\osp}{\ensuremath{\,\vee\,}}
\newcommand{\nosp}{\ensuremath{\neg}}
\newcommand{\implica}{\ensuremath{\Rightarrow}}
\newcommand{\y}{\ensuremath{\,\wedge\,}}
\renewcommand{\o}{\ensuremath{\,\vee\,}}
\newcommand{\no}{\ensuremath{\neg}}

\newcommand{\agujero}[1]{\ensuremath{<}\textit{#1}\ensuremath{>}}
\newcommand{\vble}[1]{\ensuremath{#1}}
\newcommand{\mkP}[2]{\ensuremath{p_{{#1#2}}}}

%%%%%%%%%%%%%%%%%%%%%%%%%%%%%%%%%%%%%%%%%%%%%%%%%%%
% Manejo de versiones (final o interna)
%%%%%%%%%%%%%%%%%%%%%%%%%%%%%%%%%%%%%%%%%%%%%%%%%%%

\newcommand{\final}[1]{\ifthenelse{\equal{\version}{interna}}{}{#1}}
\newcommand{\nofinal}[1]{\ifthenelse{\equal{\version}{interna}}{#1}{}}

\newcommand{\consolu}[1]{\ifthenelse{\equal{\version}{con soluciones}}{#1}{}}
\newcommand{\sinsolu}[1]{\ifthenelse{\equal{\version}{con soluciones}}{}{#1}}

\newboolean{final}
\final{\setboolean{final}{true}}
\nofinal{\setboolean{final}{false}}

\newcommand{\internal}[1]{ 
 \nofinal{\begin{flushleft} \textcolor{blue}{\upshape #1} \end{flushleft}}
}

\newcommand{\comentario}[1]{\textbf{\internal{#1}}}
\newcommand{\objetivo}[1]{\internal{\textbf{Objetivo:} #1}}

\sinsolu{\excludeversion{solucion}}
\consolu{
	\newcommand{\solucionTitle}{{\normalfont\textbf{\textcolor{blue}{Solución posible:}}}}
	\DefineVerbatimEnvironment{solucion}{Verbatim}
		{fontshape=n,tabsize=0,fontsize=\small,xleftmargin=16pt,formatcom=\solucionTitle\color{blue}}
}

\DefineVerbatimEnvironment{ejemplo}{Verbatim}{fontshape=n,tabsize=0,fontsize=\small,xleftmargin=16pt,}


\definecolor{lightgray}{gray}{0.75}

\newcommand\greybox[1]{%
  \vskip\baselineskip%
  \par\noindent\colorbox{lightgray}{%
    \begin{minipage}{\textwidth}#1\end{minipage}%
  }%
  \vskip\baselineskip%
}

\newcommand{\formalidadesPrimerEnunciado}{
  \section{Objetivos y forma de entrega}
  
  \subsection{Objetivos generales}
    El trabajo práctico debe realizarse en grupos de 5 personas. El trabajo práctico tiene como objetivo garantizar 
    que durante el transcurso de la materia todos los estudiantes pasen por la experiencia de:

    \begin{itemize}
      
      \item participar en un diseño complejo,
      \item bajar a código sus ideas de diseño en una computadora y verificar su funcionamiento mediante la utilización 
      de test cases.
      \item participar de una experiencia de trabajo grupal basada en herramientas de colaboración de tipo industrial, en
      particular compartiendo el código mediante un repositorio de fuentes compartido.
    \end{itemize}

    Para garantizar esto habrá tareas individuales y grupales. La entrega global deberá ser una sola para todo el 
    grupo, pero identificando en donde se solicite las responsabilidades que tomó cada uno. De esta forma se logra 
    un ambiente de trabajo similar al profesional, en el que cada uno tiene sus responsabilidad individuales sobre un
    proyecto compartido y el éxito del proyecto depende del trabajo conjunto.

    La evaluación del tp tendrá dos calificaciones, una grupal y otra individual. En caso de que alguno de los 
    integrantes del grupo no demuestre haberse comprometido con el trabajo grupal se le realizará un coloquio en el
    que deberá defender la solución propuesta por él y por el grupo.

    En la primera etapa del tp tendrán preponderancia las tareas grupales, con asignaciones individuales muy
    pequeñas, mientras que en la segunda parte esto se invertirá, para tener una mayoría de responsabilidades repartidas
    en asignaciones individuales.

    \subsection{Interacción con el tutor}
      Cada grupo tendrá asignado un tutor que los guiará no sólo para ir encontrando la mejor solución posible sino
      también en cuanto al proceso de trabajo, \textbf{ambos aspectos son igualmente importantes para la materia} y
      serán tenidos en cuenta para decidir la aprobación o no del trabajo práctico.

      Consideramos muy importante interactuar frecuentemente con el tutor y no esperar a la fecha de entrega para
      realizar las consultas. \textbf{No se aceptarán entregas fuera de término}, ya que esto es contrario a las necesidades
      del aprendizaje, las fechas no están puestas por capricho sino para ayudarlos a asimilar los conocimientos de la
      mejor manera posible. Si un grupo o un integrante no puede asistir al día de la entrega deberá responsabilizarse
      de coordinar una alternativa con el tutor, \textbf{en forma previa a la fecha límite}.

    \subsection{Forma de entrega}
      Para realizar el TP y almacenar tanto el código fuente como los documentos se deberá crear un área de trabajo
      compartida basada en un repositorio con capacidad de versionado, por ejemplo se pueden utilizar los repositorios
      \emph{svn} provistos en \texttt{www.xp-dev.com}.

      En cada entrega se deberá contar con el código en el aula para poder validarla. Quienes tengan la posibilidad de
      traerla en su propia notebook podrán hacerlo, en caso contrario deberán coordinar con el tutor para que él pueda
      tener el código en su propia notebook o el mecanismo alternativo que decidan en conjunto con el tutor. En caso
      que no contar con el código para evaluarlo en el momento de la entrega, la misma se considerará \emph{desaprobada}.
      
      La entrega constará del código fuente que resuelva los requerimientos y la documentación en papel correspondiente que
      explique las decisiones de diseño tomadas para llegar a dicha solución. Lo interesante de esta documentación no es una
      explicación del código (eso se puede ver allí) sino las alternativas que consideraron y los motivos por los cuales eligieron
      la solución que se implementó.


  \section{Requerimientos del Sistema}

}



\newcommand{\formalidadesSegundoEnunciado}{
 \section{Objetivos y forma de entrega}
 
   En esta siguiente parte del trabajo práctico se presentan 5 nuevos requerimientos, que se agregan a lo ya realizado
   para la anterior parte. Estos nuevos requerimientos deberán repartirse entre los integrantes del grupo, uno para
   cada uno. En los casos en los que el grupo tenga menos de 5 integrantes se podrá dejar afuera algún requerimiento,
   previo acuerdo con el tutor del grupo.

   Cabe recordar que no por esto la entrega deja de ser grupal. La entrega del TP debe ser una sola, pero
   identificando las responsabilidades asumidas por cada uno. Esto tiene dos objetivos: el primero es acercarse a la
   forma de colaboración que uno tendría en un proyecto laboral; el segundo es garantizar que todos trabajen, pero
   sin perder la posibilidad de interactuar con los compañeros.

   De esta forma cada uno deberá proponer el diseño que soluciona la tarea que le fue asignada, pero al mismo
   tiempo deberá consensuar con los demás integrantes del grupo las modificaciones que quiere hacer sobre el diseño
   base.
    \smallskip

    Para la entrega, deberán traer la documentación impresa actualizada que explica las decisiones de diseño que llevaron
    a la solución actual, así como todo el código terminado. Lo importante del documento impreso es que explique cuales
    fueron las decisiones de diseño que se tomaron para llegar a la solución. La implementación debe incluir todos los tests cases que sean necesarios que sean necesarios y que muestren el uso del sistema.

    En todos los casos nos planteamos como objetivo resolver las correcciones del TP en el día para ello deberán
    coordinar con el tutor algún mecanismo para tener a mano el código y poder verlo en conjunto.


  \section{Nuevos requerimientos}
 
}

\newcommand{\formalidadesTercerEnunciado}{
 \section{Objetivos y forma de entrega}
 
  El objetivo de esta entrega es construir un DSL utilizando el lenguaje Groovy para permitir la creación de reuniones según lo
  especificado en la siguiente sección. Es un requerimiento básico que el DSL generado interactúe con la solución desarrollada en Java
  en las entregas anteriores.
  Es importante también tratar de lograr que el DSL creado sea lo más expresivo posible considerando al menos los elementos del
  lenguaje vistos en clase.
  \smallskip
  
  Nos planteamos como objetivo resolver las correcciones del TP en el día para ello deberán coordinar con el tutor algún mecanismo
  para tener a mano el código y poder verlo en conjunto.


  \section{Nuevos requerimientos}
 
}

\title{Técnicas Avanzadas de Programación -- UTN -- FRBA\vspace{.2\baselineskip}
       \\ \cuatrimestre\
       \\ Trabajo Práctico Cuatrimestral\
       \\ \textbf{\textsc{\nombreTP}}\
       \\ \nofinal{\large \medskip Versión \version\\}
       {\small {\urlReferencia}}
       \bigskip
      }
\date{}
\newcommand{\newest}[1]{#1} %\textbf{Cursadas$\geq$2000:} #1 }
\newcommand{\oldest}[1]{}%{#1}

\newcommand{\flecha}{->}
\newcommand{\newconcept}[1]{\emph{#1}}

\DefineShortVerb{\|}

\author{}
	

\begin{document}

\maketitle


\section{Introducción}

En el mundo Pokémon, existen centros especializados para el entrenamiento y cuidado de Pokémon cuyo entrenadores son inexpertos o tienen poco tiempo. 
Estos centros toman Pokémon de cualquier tipo y preparan un esquema de entrenamiento a medida, teniendo en cuenta los requisitos y objetivos del dueño. 
Estos objetivos pueden abarcar desde aprender ciertos ataques, hasta llevar a cierta evolución deseadas o determinado estado del pokemon.
Ya que entrenar un Pokémon es una tarea muy compleja (y puede lastimar al Pokémon), se busca implementar en Scala un programa que permita simular un posible entrenamiento antes de llevarlo a cabo, así como también elegir el mejor entrenamiento posible.
\\\\
\textbf{IMPORTANTE}: Este trabajo práctico debe implementarse de manera que se apliquen los principios del paradigma híbrido objeto-funcional enseñados en clase. No alcanza con hacer que el código funcione en objetos, hay que aprovechar las herramientas funcionales, poder justificar las decisiones de diseño y elegir el modo y lugar para usar conceptos de un paradigma u otro.
Se tendrán en cuenta para la corrección los siguientes aspectos:

\begin{itemize}
\item Uso de Inmutabilidad vs. Mutabilidad
\item Uso de Polimorfismo paramétrico (Pattern Matching) vs. Polimorfismo Ad-Hoc
\item Aprovechamiento del polimorfismo entre objetos y funciones
\item Uso adecuado de herramientas funcionales
\item Cualidades de Software
\item Diseño de interfaces y elección de tipos
\end{itemize}

\section{Descripción General del Dominio}

Los Pokémon son animalitos fantásticos que los niños atrapan, entrenan y obligan a luchar por diversión. Sobre estas criaturas se han realizado diversos estudios, que nos permiten clasificarlos, estudiarlos y entrenarlos mejor, para volverlos maquinas de matar más efectivas.
\\\\
Nuestro simulador de entrenamiento tiene el objetivo de analizar el estado físico de un Pokémon en un momento dado, así como también el impacto que podría tener en él realizar cierta actividad (o secuencia de actividades), evitando así someter a la criaturita a un sufrimiento innecesario. Para esto hace falta tener un conocimiento detallado de la información con la que contamos acerca de los Pokémon.

\subsection{Características}

De cada Pokémon conocemos un conjunto de características que pueden ser fácilmente medidas y debemos tener en consideración a la hora de entrenar:

\begin{itemize}
\item \textbf{Nivel}: Representa el crecimiento general del Pokémon y puede ser usado para estimar, a grandes rasgos, que tan fuerte es (a mayor nivel, más poderoso). Se representa con un número entre 1 y 100.
\item \textbf{Experiencia}: Es un valor numérico que indica la cantidad de experiencia acumulada a por el Pokémon a través del tiempo. El incremento en la experiencia conlleva a un incremento en el nivel.
\item \textbf{Género}: Representa la identidad sexual del Pokémon. Puede ser Macho o Hembra.
\item \textbf{Energía}: Es una medida de la salud general del Pokémon, la cual disminuye conforme este sufre daño o se esfuerza mucho y aumenta cuando descansa o se cura o se alimenta. Si la energía llega a 0, el Pokémon está en peligro…
\item \textbf{Energía Máxima}: Es el máximo valor posible para la Energía. Cualquier incremento de Energía por encima de este valor es redondeado.
\item \textbf{Peso}: Indica que tan pesado o voluminoso es un Pokémon. Está expresado en Kg. y va de 0 (algunos Pokémon son muy, muy livianos) en adelante.
\item \textbf{Fuerza}: Valor que representa la tonificación muscular o capacidad de carga. Va de 1 a 100.
\item \textbf{Velocidad}: Valor que representa la agilidad o rapidez de movimiento. Va de 1 a 100.
\end{itemize}

\subsection{Estado}

Las características de un Pokémon reflejan los rasgos generales de su estado de salud; sin embargo, existen también ciertas aflicciones que pueden afectar a un Pokémon, más allá de sus características. Los cuadros conocidos se detallan a continuación:

\begin{itemize}
\item \textbf{Dormido}: El Pokémon está profundamente dormido. Muy exhausto para hacer nada.
\item \textbf{Envenenado}: El Pokémon sufrió una intoxicación aguda y ahora le duele todo.
\item \textbf{Paralizado}: El Pokémon no puede moverse en absoluto.
\item \textbf{K.O.}: El Pokémon quedó inconsciente, al borde de la muerte.
\end{itemize}

Hay que tener en cuenta que un Pokémon solo puede sufrir uno de estos cuadros a la vez. 
La forma en que los diversos estados impactan en la realización de actividades y las posibles transiciones de un estado al otro se describen en detalle más adelante.

\subsection{Especie}

Todos los Pokémon pertenecen a alguna de las 151\footnote{Sí. 151 especies. Todo lo que vino después es horrible y no existió. LALALALALALA} Especies\footnote{http://www.pokemon.com/es/pokedex/} conocidas. 
La Especie a la que un Pokémon pertenece es determinante a la hora de calcular su progreso, cómo reaccionará su cuerpo ante cierta actividad, o si puede o no aprender un ataque.
\\\\
Cuando un pokémon sube de nivel, recibe un incremento en sus carácteristicas Energía Máxima, Peso, Fuerza y Velocidad que depende de su especie. 
De cada especie se conoce cuánto incrementan las características de sus miembros con cada subida de nivel; además, la cantidad de Experiencia necesaria para subir al próximo nivel depende de la \textbf{ Resistencia Evolutiva } de su Especie (ver más adelante).
\\\\
También cada especie determina un \textbf{Peso Máximo}, que indica que tanto Peso pueden ganar sus miembros antes de que afecte su salud.
\\\\
De cada especie se conoce además su \textbf{Tipo}. 
El Tipo representa la afinidad de los miembros de una Especie a cierto elemento o naturaleza y su destreza natural (o no) para realizar ciertas actividades. 
Todas las especies poseen un \textit{ Tipo } \textit{"Principal"}, sin embargo, algunas especies poseen también un Tipo \textit{"Secundario"}, que denota un menor grado de afinidad a otra naturaleza.

\greybox{Por ejemplo, la especie \textbf{ Zapdos } es de tipo \textbf{ Eléctrico/Volador}, mientras que la especie \textbf{ Pikachu } es sólo de tipo \textbf{ Eléctrico}.}

\subsection{Tipos}

Los \textit{ Tipos } forman parte de un sistema de clasificación basado en la afinidad a cierto poder o elemento natural. 
Varias cosas en el mundo de Pokémon (cómo las Especies, los Ataques y algunos objetos) pueden verse asociadas a uno o más \textit{ Tipos}.
\\\\
Dado que los \textit{ Tipos } representan la influencia de una fuerza natural, es posible afirmar que ciertos tipos son superiores a otros, basándose en cómo estas fuerzas se influencian entre sí. De este modo se puede determinar si un \textit{ Tipos } "Gana" o "Pierde" contra otro. 
La relación entre los distintos \textit{ Tipos } se ve en la siguiente tabla:

\begin{center}
    \begin{tabular}{ | l | l |}
    \hline
    Tipo & Le Gana A... \\ \hline
    Fuego & Planta, Hielo, Bicho.\\ \hline
    Agua & Fuego, Tierra, Roca \\ \hline
    Planta & Agua, Tierra, Roca.\\ \hline
    Tierra & Fuego, Eléctrico, Veneno, Roca.\\ \hline
    Hielo & Planta, Tierra, Volador, Dragón.\\ \hline
    Roca & Fuego, Hielo, Volador, Bicho.\\ \hline
    Eléctrico & Agua, Volador.\\ \hline
    Psíquico & Pelea, Veneno.\\ \hline
    Pelea & Normal, Hielo, Roca.\\ \hline
    Fantasma & Psíquico, Fantasma.\\ \hline
    Volador & Planta, Pelea, Bicho.\\ \hline
    Bicho & Planta, Psiquico.\\ \hline
    Veneno & Planta.\\ \hline
    Dragón & Dragón.\\ \hline
    Normal & \\ \hline
    \end{tabular}
\end{center}

\greybox{Por ejemplo, el tipo \textbf{ Psíquico } es bueno contra los tipos \textbf{ Pelea} y \textbf{ Veneno}, pero es débil contra el tipo \textbf{ Fantastama}.}

\subsection{Ataques}

Cada Pokémon conoce al menos un movimiento o técnica que puede usar durante una pelea, estos movimientos se denominan \textit{ Ataques}.
\\\\
Cada Ataque pertenece a \textbf{ un único Tipo} que representa su afinidad a alguna fuerza natural (similar a lo que ocurre con las Especies). 
Un Pokémon sólo puede aprender ataques de un tipo afín a su Especie, como se describe más adelante.
\\\\
Una vez que un Pokémon aprende un \textit{ Ataque } puede usarlo una cierta cantidad de veces antes de tener que descansar. 
El número de veces que un Ataque puede ser usado antes de descansar se denomina \textbf{ Puntos de Ataque}. 
Cada vez que un Pokémon usa un Ataque pierde un PA; si ya no le quedan PAs para ese ataque, entonces \textbf{ no puede} atacar.
\\\\
El \textit{ Máximo} inicial de \textbf{Puntos de Ataque} que un Pokemón recibe al aprender un \textit{Ataque} depende de cada \textit{Ataque}, sin embargo, es posible, 
con un entrenamiento riguroso, aumentar el número \textit{Máximo} de \textit{Puntos de Ataque} que un Pokémon tiene disponible.

\greybox{Por ejemplo, el ataque de tipo \textbf{Normal} llamado \textbf{Mordida} tiene un Máximo de 30 \textbf{P.A.} 
Entonces, cuando un Pokémon lo aprende, empieza con 30 \textbf{P.A.} disponibles y perderá 1 \textbf{P.A.} cada vez que use el ataque. 
El Pokémon puede recuperar \textbf{P.A.} pero nunca por encima de su \textbf{Máximo}; sin embargo, es posible para ese Pokémon entrenarse y subir su Máximo a un número mayor, mientras que el de todos los demás queda en 30.}

Al efectuar un \textit{Ataque} el Pokémon podría sufrir un efecto colateral que depende de cada ataque.

\greybox{
\textit{Algunos ejemplos posibles de Efectos son:\newline
El ataque \textbf{Reposar} aumenta la energía del Pokémon al máximo, pero lo deja dormido.\newline
El ataque \textbf{Enfocarse} sube la velocidad del Pokémon un punto.\newline
El ataque \textbf{Endurecerse} sube la energía del Pokémon 5 puntos, pero lo paraliza.}}

\subsection{Evolución}

Tras haber satisfecho cierto requerimiento, los Pokémon cambian de especie. 
A esto se le llama \textit{"Evolucionar"}. De cada especie se sabe cuál otra especie es su evolución y la condición que sus miembros deben cumplir para evolucionar. 
Tener en cuenta que \textbf{no todas} las especies pueden evolucionar y, aquellas que sí, tienen \textbf{una única} posible evolución
\\\\
Las condiciones que cada Especie puede tener para evolucionar son las siguientes:
\\\\

\begin{itemize}
\item \textbf{Subir de Nivel}: Los miembros de las \textit{Especies} con esta condición evolucionan a otra especie cuando alcanzan un cierto nivel, que puede variar para cada \textit{Especie}.
\item \textbf{Intercambiar}: Los miembros de \textit{Especies} con esta condición sólo evolucionan como consecuencia del profundo trauma emocional que sufren si creen que sus dueños los han \textit{"Intercambiado"} por otro Pokémon.
\item \textbf{Usar Piedra}: Los miembros de \textit{Especies} con esta condición sólo evolucionan cuando son expuestos a la radiación de unos objetos conocidos como "Piedras Evolutivas". 
Cada Piedra tiene asociado un Tipo que debe coincidir con el \textit{Tipo Principal} de la Especie para gatillar la evolución. La única excepción a esto son las \textit{Piedras Lunares}, que hacen evolucionar Pokémon de especies arbitrarias.
\end{itemize}

\greybox{
Ej:\newline
La especie \textbf{Charmander} evoluciona a \textbf{Charmeleon} al llegar al Nivel 16, la cual a su vez, evoluciona en \textbf{Charizard} al llegar al Nivel 36. La especie \textbf{Charizard} no evoluciona.\newline
La especie \textbf{Machop} evoluciona en Machoke al llegar al Nivel 28, pero un \textbf{Machoke} evoluciona en \textbf{Machamp} al ser intercambiado.\newline
La especie \textbf{Vulpix} evoluciona en \textbf{Ninetales} al ser expuesto a una Piedra Fuego, mientras que la especie \textbf{Jigglypuff} evoluciona en \textbf{Wigglytuff} usando una Piedra Lunar.\newline
La especie \textbf{Electabuzz} no evoluciona nunca.}

Cuando un Pokémon evoluciona los incrementos que gana por subir de Nivel se aplican de forma retroactiva. Esto quiere decir que el Pokémon incrementa sus Características como si siempre hubiera sido de la nueva Especie.

\section{Requerimientos}

Se pide implementar los siguientes casos de uso, acompañados de sus correspondientes tests y la documentación necesaria para defender su diseño (MÍNIMO un diagrama de clases):

\subsection*{1.Ganar experiencia}

El sistema debe poder registrar el "progreso" de un Pokémon y la actualización de sus estado general.
\\\\
Ciertas \textit{Actividades} pueden hacer que el Pokémon gane alguna cantidad de \textit{Experiencia}. Cuando esto ocurre, el Pokémon puede experimentar diversos cambios.
\\\\
Si la \textit{Experiencia} alcanza en valor necesario para subir al siguiente Nivel, este aumenta en 1 y las \textit{Características} del Pokémon incrementan de acuerdo a los valores correspondientes a su \textit{Especie}.
En este punto debe llevarse a cabo, si corresponde, la evolución del Pokémon.
\\\\
La cantidad de \textit{Experiencia} para llegar a un nivel es el doble de la necesaria para el nivel anterior, sumado a la Resistencia Evolutiva de la Especie:
\\\\
Experiencia para Nivel N = 2 * Experiencia para Nivel (N - 1) + Resistencia Evolutiva
\\\\
Obviamente, siendo que todos los Pokémon son, mínimo, \textit{Nivel} 1, no es necesaria ninguna Experiencia para llegar a este nivel.

\greybox{\textit{Por ej, un \textbf{Charmander} tiene una resistencia evolutiva de 350. Eso significa que va a pasar del nivel 1 al 2 cuando junte 350 puntos de experiencia y al nivel 3 cuando junte 1050 (2 * 350 + 350) y al nivel 4 cuando junte 2450 (2 * 1050 + 350).}}

\textbf{NOTA:} Hay que tratar de modelar esto con la mejor representación y menor cantidad de redundancia posible.

\subsection*{2. Actividades}

En el Gimnasio cuentan con un repertorio de \textit{Actividades} para que los Pokémon realicen a modo de entrenamiento. 
Llevar a cabo una \textit{Actividad} puede repercutir de muchas formas diversas y el sistema debe poder anticiparse a estos cambios, prediciendo cómo puede verse afectado el Pokémon entrenado. 
Por esta razón, se desea poder calcular cómo quedaría el Pokémon al realizar una \textit{Actividad}.
\\\\
Hay que tener en cuenta que un Pokémon no puede realizar ninguna Actividad si se encuentra \textit{K.O.}, así como tampoco puede realizar \textit{Actividades} que dejarían sus \textit{Características} en valores inválidos (Ej: Fuerza > 100 o Peso < 0).
\\\\
También hay que considerar que un Pokémon Dormido "ignora" las \textit{Actividades} que se le asignan, es decir, acepta la orden, pero no hace nada y, por lo tanto, no sufre ningún cambio. 
Un Pokémon dormido puede "ignorar" hasta 3 \textit{Actividades}; después de eso se despierta y se comporta normalmente.
\\\\
Las Actividades conocidas son las siguientes:
\\\\
\textbf{Realizar un Ataque:}\newline
Cuando se le pide a un Pokémon que realice un \textit{Ataque} en particular, este gasta un \textit{Punto de Ataque} para ese \textit{Ataque} y gana \textit{Experiencia}. 
La Experiencia ganada depende de varios factores:
\begin{itemize}
 \item Si el Ataque es del Tipo Principal del Pokémon, gana 50 puntos.
 \item Si el Ataque es del Tipo Secundario del Pokémon y este es Macho, gana 20 puntos.
 \item Si el Ataque es del Tipo Secundario del Pokémon y este es Hembra, gana 40 puntos (porque es un hecho conocido que las nenas aprenden más rápido).
 \item Los Ataques de tipo Dragón son especialmente difíciles de hacer, así que ignoran las reglas anteriores y siempre dan 80 puntos.
\end{itemize}

Además de estos cambios, el Pokémon debe sufrir el efecto secundario del \textit{ Ataque }, si es que tiene alguno.
En caso de que el Pokémon no conociera el \textit{Ataque} o no tuviera suficientes Puntos de Ataque NO PUEDE realizar la \textit{Actividad}.\newline
\\\\
\textbf{Levantar Pesas:}\newline
Cuando un Pokémon levanta pesas, gana 1 punto de experiencia por cada kilo levantado.\newline
Los Pokémon con Tipo Principal o Secundario Pelea ganan el doble de puntos.\newline
Los Pokémon de tipo Fantasma NO PUEDEN levantar pesas.\newline
Si un Pokémon levanta más de 10 kilos por cada punto de Fuerza, no gana nada de Experiencia y queda Paralizado.\newline
Si un Pokémon Paralizado levanta pesas, no gana nada de Experiencia y queda K.O..\newline
\\\\

\textbf{Nadar}:\newline
Por cada minuto de nado, un Pokémon pierde 1 punto de Energía y gana 200 de Experiencia.\newline
Los Pokémon de Tipo Agua ganan, además, 1 punto de Velocidad por hora.\newline
Los Pokémon con un Tipo Principal o Secundario que "pierde" contra el Tipo Agua no ganan nada de Experiencia y quedan K.O. automáticamente.
\\\\
\textbf{Aprender Ataque:}\newline
Cuando se le pide a un Pokémon que aprenda un cierto Ataque, hay dos resultados posibles:
\begin{itemize}
 \item Si el Ataque es de un Tipo "afín" a su \textit{Especie}, el Pokémon incorpora el \textit{Ataque} y adquiere automáticamente una cantidad de \textit{Puntos de Ataque} igual al Máximo de Puntos del Ataque.
Se dice que un Tipo es afín a una Especie si es el Tipo "Normal" o es el \textit{Tipo Principal} o \textit{Secundario} de la Especie.
 \item Si el Ataque NO es de un \textit{Tipo} afín el Pokémon se lastima tratando de aprenderlo, no aprende nada y queda K.O..
\end{itemize}

\textbf{Usar Piedra:}\newline
Si un Pokémon cuya especie tiene como Condición Evolutiva Usar una Piedra Lunar usa una Piedra Lunar, evoluciona.\newline
Si un Pokémon cuya especie tiene "Usar Piedra" cómo Condición Evolutiva usa una Piedra del mismo Tipo que el Tipo Principal de su Especie, evoluciona.\newline
En cualquier otro caso, si Tipo de la Piedra le "gana" al Tipo Principal o Secundario de la Especie la radiación de la Piedra lo deja Envenenado.
\\\\
\textbf{Usar Poción:}\newline
Al usar una Poción el Pokémon se cura hasta un máximo de 50 puntos de Energía.
\\\\
\textbf{Usar Antidoto:}\newline
Al usar un Antidoto un Pokémon Envenenado deja de estarlo. De otro modo no pasa nada.
\\\\
\textbf{Usar Ether:}\newline
Al usar un Ether un Pokémon se cura de cualquier estado, menos K.O.. De otro modo no pasa nada.
\\\\
\textbf{Comer Hierro:}\newline
Comer Hierro aumenta en 5 la Fuerza de un Pokémon.
\\\\
\textbf{Comer Calcio:}\newline
Comer Calcio aumenta en 5 la Velocidad de un Pokémon.
\\\\
\textbf{Comer Zinc:}\newline
Comer Zinc aumenta en 2 el Máximo de Puntos de Ataque de todos los Ataques del Pokémon.
\\\\
\textbf{Descansar:}\newline
Cuando un Pokémon descansa recupera al máximo los Puntos de Ataque de todos sus Ataques.
Además, un Pokémon que no sufre de ningún Estado y tiene menos del 50\% de Energía, se queda Dormido.
\\\\
\textbf{Fingir Intercambio:}\newline
Los Pokémon cuya Especie tiene como Condición Evolutiva "Intercambiar", evolucionan.
Los demás sólo se ponen tristes y, si son Macho, suben 1 kilo de Peso y si son Hembra bajan 10.
\\\\
\textbf{NOTA:} Es importante diferenciar el caso en que un Pokémon "NO PUEDE" hacer una actividad del caso en que la hace y no sufre cambios. 
Es \textbf{MUY IMPORTANTE} que cuando le pido a un Pokémon que haga algo \textbf{LO HAGA} o me informe de alguna manera que no pudo.

\subsection*{2.b (BONUS)}
Implementar el punto 2 \textbf{SIN DEFINIR NUEVAS CLASSES/OBJECTS} para las distintas actividades.

\subsection*{3. Rutinas}

Queremos ahora crear Rutinas, que son simplemente conjuntos ordenados de Actividades que un Pokémon puede realizar, a los que damos un nombre.
De la misma manera que con las Actividades, nuestro sistema debe permitir consultar  cómo queda un Pokémon después de realizar una Rutina.
Es necesario tener en cuenta que no necesariamente todo Pokémon es capaz de realizar cualquier Rutina y, en caso de que no pueda, sería importante poder averiguar porqué.

\subsection*{4. Analisis de Rutinas}

Dado un Pokémon y un conjunto de Rutinas queremos que el sistema pueda encontrar el nombre de la mejor Rutina (si es que existe alguna...) para ese Pokémon, de acuerdo a un criterio que nos pida el dueño.
\\\\
No estamos seguros de que criterios nos pedirá aplicar el dueño, por lo que debemos crear un diseño lo más flexible posible. 
Sabemos que algunos \textbf{ejemplos de criterios} son:

\begin{itemize}
 \item Elegir la Rutina que lo haga subir al mayor Nivel posible
 \item Elegir la Rutina que lo deje con la mayor cantidad de energía posible.
 \item Elegir la Rutina que lo vuelva lo menos pesado posible.
\end{itemize}

Es especialmente importante considerar que podría no haber ninguna Rutina en el conjunto que capaz de ser realizada por el Pokémon en cuestión, y el sistema debe manejar esto de forma acorde.


\end{document}


