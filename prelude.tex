%%!TEX encoding = UTF-8 Unicode 
\usepackage[utf8]{inputenc}
\usepackage{amstext}
\usepackage{ifthen}
\usepackage{graphicx}
\usepackage{array}
\usepackage{xspace}
\usepackage{color}
\usepackage{fancyvrb}
\usepackage{version}
\usepackage{theorem}
\usepackage{listings}
\usepackage{hyperref}

\usepackage[spanish]{babel}

\textwidth 17.5cm
\textheight 26cm
\oddsidemargin -1cm
%\evensidemargin -.3cm
\topmargin -1cm
\headheight 0cm
\headsep 0cm

%\textwidth 16cm
%\textheight 25cm
%\oddsidemargin -.64cm
%\evensidemargin -.64cm
%\topmargin -1.5cm
%\headheight 0cm
%\headsep 0cm
\definecolor{gray}{rgb}{0.5,0.5,0.5}

% "define" Scala
\lstdefinelanguage{scala}{
  morekeywords={abstract,case,catch,class,def,%
    do,else,extends,false,final,finally,%
    for,if,implicit,import,match,mixin,%
    new,null,object,override,package,%
    private,protected,requires,return,sealed,%
    super,this,throw,trait,true,try,%
    type,val,var,while,with,yield},
  otherkeywords={=>,<-,<\%,<:,>:,\#,@},
  sensitive=true,
  morecomment=[l]{//},
  morecomment=[n]{/*}{*/},
  morestring=[b]",
  morestring=[b]',
  morestring=[b]"""
}

\lstset{ %
  backgroundcolor=\color{white},  % choose the background color. You must add \usepackage{color}
  showspaces=false,               % show spaces adding particular underscores
  showstringspaces=false,         % underline spaces within strings
  showtabs=false,                 % show tabs within strings adding particular underscores
  frame=single,	                % adds a frame around the code
  tabsize=2,	                % sets default tabsize to 2 spaces
  captionpos=true,                   % sets the caption-position to bottom
  breaklines=true,                % sets automatic line breaking
  breakatwhitespace=true,        % sets if automatic breaks should only happen at whitespace
  escapeinside={\%*}{*)/},         % if you want to add a comment within your code
  aboveskip=3mm,
  belowskip=3mm,
  columns=flexible,
  basicstyle=\small\ttfamily,
  numberstyle=\tiny\color{gray},
  keywordstyle=\bfseries,
  commentstyle=\color{gray},
  stringstyle=\color{gray},
}

\lstdefinelanguage{smalltalk}{
  morekeywords={self,super},
  otherkeywords={:=,^,\[,\],>>,\(,\),\;},
  sensitive=true,
  morecomment=[n]{"},
  morestring=[b]'
}

\definecolor{dkgreen}{rgb}{0,0.6,0}
\definecolor{gray}{rgb}{0.5,0.5,0.5}
\definecolor{mauve}{rgb}{0.58,0,0.82}
 
% Default settings for code listings
\lstset{frame=tb,
  language=scala,
  aboveskip=3mm,
  belowskip=3mm,
  showstringspaces=false,
  columns=flexible,
  basicstyle={\small\ttfamily},
  numbers=none,
  numberstyle=\tiny\color{gray},
  keywordstyle=\color{blue},
  commentstyle=\color{dkgreen},
  stringstyle=\color{mauve},
  frame=single,
  breaklines=true,
  breakatwhitespace=true
  tabsize=3
}

\renewcommand{\thepage}{}
\newcommand{\ignore}[1]{}
\newcommand{\mat}[1]{\ensuremath{#1}}

\newenvironment{display}
   {\begin{list}{}{\setlength{\topsep}{0cm}
                   \setlength{\leftmargin}{0cm}
                  }}{\end{list}}

\newcommand{\textfol}[1]{\ensuremath{\textsf{#1}}}
\newcommand{\textcode}[1]{{\normalfont{\texttt{#1}}}}
\newcommand{\propername}[1]{\textsc{#1}}

%\renewcommand{\labelenumi}{\textbf{\alph{enumi})}}
\renewcommand{\labelenumii}{\textbf{\alph{enumii})}}

\newcommand{\pred}[2]{\ensuremath{\text{\textbf{#1}}(#2)}}
\newcommand{\predName}[1]{\ensuremath{\text{\textbf{#1}}}}
\newcommand{\var}[1]{\ensuremath{#1}}
\newcommand{\code}[1]{\texttt{#1}}

\newcommand{\paratodo}[2]{\ensuremath{(\forall{#1})#2}}
\newcommand{\existe}[2]{\ensuremath{(\exists{#1})#2}}
\newcommand{\implicasp}{\ensuremath{\Rightarrow}}
\newcommand{\ysp}{\ensuremath{\,\wedge\,}}
\newcommand{\osp}{\ensuremath{\,\vee\,}}
\newcommand{\nosp}{\ensuremath{\neg}}
\newcommand{\implica}{\ensuremath{\Rightarrow}}
\newcommand{\y}{\ensuremath{\,\wedge\,}}
\renewcommand{\o}{\ensuremath{\,\vee\,}}
\newcommand{\no}{\ensuremath{\neg}}

\newcommand{\agujero}[1]{\ensuremath{<}\textit{#1}\ensuremath{>}}
\newcommand{\vble}[1]{\ensuremath{#1}}
\newcommand{\mkP}[2]{\ensuremath{p_{{#1#2}}}}

%%%%%%%%%%%%%%%%%%%%%%%%%%%%%%%%%%%%%%%%%%%%%%%%%%%
% Manejo de versiones (final o interna)
%%%%%%%%%%%%%%%%%%%%%%%%%%%%%%%%%%%%%%%%%%%%%%%%%%%

\newcommand{\final}[1]{\ifthenelse{\equal{\version}{interna}}{}{#1}}
\newcommand{\nofinal}[1]{\ifthenelse{\equal{\version}{interna}}{#1}{}}

\newcommand{\consolu}[1]{\ifthenelse{\equal{\version}{con soluciones}}{#1}{}}
\newcommand{\sinsolu}[1]{\ifthenelse{\equal{\version}{con soluciones}}{}{#1}}

\newboolean{final}
\final{\setboolean{final}{true}}
\nofinal{\setboolean{final}{false}}

\newcommand{\internal}[1]{ 
 \nofinal{\begin{flushleft} \textcolor{blue}{\upshape #1} \end{flushleft}}
}

\newcommand{\comentario}[1]{\textbf{\internal{#1}}}
\newcommand{\objetivo}[1]{\internal{\textbf{Objetivo:} #1}}

\sinsolu{\excludeversion{solucion}}
\consolu{
	\newcommand{\solucionTitle}{{\normalfont\textbf{\textcolor{blue}{Solución posible:}}}}
	\DefineVerbatimEnvironment{solucion}{Verbatim}
		{fontshape=n,tabsize=0,fontsize=\small,xleftmargin=16pt,formatcom=\solucionTitle\color{blue}}
}

\DefineVerbatimEnvironment{ejemplo}{Verbatim}{fontshape=n,tabsize=0,fontsize=\small,xleftmargin=16pt,}


\definecolor{lightgray}{gray}{0.75}

\newcommand\greybox[1]{%
  \vskip\baselineskip%
  \par\noindent\colorbox{lightgray}{%
    \begin{minipage}{\textwidth}#1\end{minipage}%
  }%
  \vskip\baselineskip%
}

\newcommand{\formalidadesPrimerEnunciado}{
  \section{Objetivos y forma de entrega}
  
  \subsection{Objetivos generales}
    El trabajo práctico debe realizarse en grupos de 5 personas. El trabajo práctico tiene como objetivo garantizar 
    que durante el transcurso de la materia todos los estudiantes pasen por la experiencia de:

    \begin{itemize}
      
      \item participar en un diseño complejo,
      \item bajar a código sus ideas de diseño en una computadora y verificar su funcionamiento mediante la utilización 
      de test cases.
      \item participar de una experiencia de trabajo grupal basada en herramientas de colaboración de tipo industrial, en
      particular compartiendo el código mediante un repositorio de fuentes compartido.
    \end{itemize}

    Para garantizar esto habrá tareas individuales y grupales. La entrega global deberá ser una sola para todo el 
    grupo, pero identificando en donde se solicite las responsabilidades que tomó cada uno. De esta forma se logra 
    un ambiente de trabajo similar al profesional, en el que cada uno tiene sus responsabilidad individuales sobre un
    proyecto compartido y el éxito del proyecto depende del trabajo conjunto.

    La evaluación del tp tendrá dos calificaciones, una grupal y otra individual. En caso de que alguno de los 
    integrantes del grupo no demuestre haberse comprometido con el trabajo grupal se le realizará un coloquio en el
    que deberá defender la solución propuesta por él y por el grupo.

    En la primera etapa del tp tendrán preponderancia las tareas grupales, con asignaciones individuales muy
    pequeñas, mientras que en la segunda parte esto se invertirá, para tener una mayoría de responsabilidades repartidas
    en asignaciones individuales.

    \subsection{Interacción con el tutor}
      Cada grupo tendrá asignado un tutor que los guiará no sólo para ir encontrando la mejor solución posible sino
      también en cuanto al proceso de trabajo, \textbf{ambos aspectos son igualmente importantes para la materia} y
      serán tenidos en cuenta para decidir la aprobación o no del trabajo práctico.

      Consideramos muy importante interactuar frecuentemente con el tutor y no esperar a la fecha de entrega para
      realizar las consultas. \textbf{No se aceptarán entregas fuera de término}, ya que esto es contrario a las necesidades
      del aprendizaje, las fechas no están puestas por capricho sino para ayudarlos a asimilar los conocimientos de la
      mejor manera posible. Si un grupo o un integrante no puede asistir al día de la entrega deberá responsabilizarse
      de coordinar una alternativa con el tutor, \textbf{en forma previa a la fecha límite}.

    \subsection{Forma de entrega}
      Para realizar el TP y almacenar tanto el código fuente como los documentos se deberá crear un área de trabajo
      compartida basada en un repositorio con capacidad de versionado, por ejemplo se pueden utilizar los repositorios
      \emph{svn} provistos en \texttt{www.xp-dev.com}.

      En cada entrega se deberá contar con el código en el aula para poder validarla. Quienes tengan la posibilidad de
      traerla en su propia notebook podrán hacerlo, en caso contrario deberán coordinar con el tutor para que él pueda
      tener el código en su propia notebook o el mecanismo alternativo que decidan en conjunto con el tutor. En caso
      que no contar con el código para evaluarlo en el momento de la entrega, la misma se considerará \emph{desaprobada}.
      
      La entrega constará del código fuente que resuelva los requerimientos y la documentación en papel correspondiente que
      explique las decisiones de diseño tomadas para llegar a dicha solución. Lo interesante de esta documentación no es una
      explicación del código (eso se puede ver allí) sino las alternativas que consideraron y los motivos por los cuales eligieron
      la solución que se implementó.


  \section{Requerimientos del Sistema}

}



\newcommand{\formalidadesSegundoEnunciado}{
 \section{Objetivos y forma de entrega}
 
   En esta siguiente parte del trabajo práctico se presentan 5 nuevos requerimientos, que se agregan a lo ya realizado
   para la anterior parte. Estos nuevos requerimientos deberán repartirse entre los integrantes del grupo, uno para
   cada uno. En los casos en los que el grupo tenga menos de 5 integrantes se podrá dejar afuera algún requerimiento,
   previo acuerdo con el tutor del grupo.

   Cabe recordar que no por esto la entrega deja de ser grupal. La entrega del TP debe ser una sola, pero
   identificando las responsabilidades asumidas por cada uno. Esto tiene dos objetivos: el primero es acercarse a la
   forma de colaboración que uno tendría en un proyecto laboral; el segundo es garantizar que todos trabajen, pero
   sin perder la posibilidad de interactuar con los compañeros.

   De esta forma cada uno deberá proponer el diseño que soluciona la tarea que le fue asignada, pero al mismo
   tiempo deberá consensuar con los demás integrantes del grupo las modificaciones que quiere hacer sobre el diseño
   base.
    \smallskip

    Para la entrega, deberán traer la documentación impresa actualizada que explica las decisiones de diseño que llevaron
    a la solución actual, así como todo el código terminado. Lo importante del documento impreso es que explique cuales
    fueron las decisiones de diseño que se tomaron para llegar a la solución. La implementación debe incluir todos los tests cases que sean necesarios que sean necesarios y que muestren el uso del sistema.

    En todos los casos nos planteamos como objetivo resolver las correcciones del TP en el día para ello deberán
    coordinar con el tutor algún mecanismo para tener a mano el código y poder verlo en conjunto.


  \section{Nuevos requerimientos}
 
}

\newcommand{\formalidadesTercerEnunciado}{
 \section{Objetivos y forma de entrega}
 
  El objetivo de esta entrega es construir un DSL utilizando el lenguaje Groovy para permitir la creación de reuniones según lo
  especificado en la siguiente sección. Es un requerimiento básico que el DSL generado interactúe con la solución desarrollada en Java
  en las entregas anteriores.
  Es importante también tratar de lograr que el DSL creado sea lo más expresivo posible considerando al menos los elementos del
  lenguaje vistos en clase.
  \smallskip
  
  Nos planteamos como objetivo resolver las correcciones del TP en el día para ello deberán coordinar con el tutor algún mecanismo
  para tener a mano el código y poder verlo en conjunto.


  \section{Nuevos requerimientos}
 
}

\title{Técnicas Avanzadas de Programación -- UTN -- FRBA\vspace{.2\baselineskip}
       \\ \cuatrimestre\
       \\ Trabajo Práctico Cuatrimestral\
       \\ \textbf{\textsc{\nombreTP}}\
       \\ \nofinal{\large \medskip Versión \version\\}
       {\small {\urlReferencia}}
       \bigskip
      }
\date{}
\newcommand{\newest}[1]{#1} %\textbf{Cursadas$\geq$2000:} #1 }
\newcommand{\oldest}[1]{}%{#1}

\newcommand{\flecha}{->}
\newcommand{\newconcept}[1]{\emph{#1}}

\DefineShortVerb{\|}

\author{}
	
